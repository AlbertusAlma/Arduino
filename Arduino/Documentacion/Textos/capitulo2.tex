\chapter{Estructura de un programa}

La estructura básica del lenguaje de programación de Arduino es bastante simple y se compone de al menos dos partes. Estas dos partes necesarias, o funciones, encierran bloques que contienen declaraciones, estamentos o instrucciones.
\begin{lstlisting}
void setup() //Primera Parte
{
estamentos;
}
void loop() //Segunda Parte
{
estamentos;
}
\end{lstlisting}
 En donde \textbf{setup()} es la parte encargada de recoger la configuración y \textbf{loop()} es la que contiene el programa que se ejecutará cíclicamente (de ahí el término loop –bucle-). Ambas funciones son necesarias para que el programa trabaje.\\
La función de configuración (setup) debe contener la declaración de las variables. Es la primera función a ejecutar en el programa, se ejecuta sólo una vez, y se utiliza para configurar o inicializar pinMode (modo de trabajo de las E/S), configuración de la comunicación en serie y otras.\\
La función bucle (loop) siguiente contiene el código que se ejecutara continuamente (lectura de entradas, activación de salidas, etc) Esta función es el núcleo de todos los programas de Arduino y la que realiza la mayor parte del trabajo.
\section{setup()}
La función setup() se invoca una sola vez cuando el programa empieza. Se utiliza para inicializar los modos de trabajo de los pins, o el puerto serie. Debe ser incluido en un programa aunque no haya declaración que ejecutar. Así mismo se puede utilizar para establecer el estado inicial de las salidas de la placa.
\begin{lstlisting}
void setup()
{
pinMode(pin, OUTPUT); // configura el pin como salida
digitalWrite(pin, HIGH);  // pone el pin en estado HIGH
}
\end{lstlisting}
\section{loop()}
Después de llamar a setup(), la función loop() hace precisamente lo que sugiere su nombre, se ejecuta de forma cíclica, lo que posibilita que el programa esté respondiendo continuamente ante los eventos que se produzcan en la placa.
\begin{lstlisting}
void loop()
{
digitalWrite(pin, HIGH); // pone en uno (on, 5v) el pin
delay(1000);            	// espera un segundo (1000 ms)
digitalWrite(pin, LOW); // pone en cero (off, 0v.) el pin
delay(1000);
}
\end{lstlisting}
\section{Funciones}
Una función es un bloque de código que tiene un nombre y un conjunto de instrucciones que son ejecutadas cuando se llama a la función. Son funciones setup() y loop() de las que ya se ha hablado. Las funciones de usuario pueden ser escritas para realizar tareas repetitivas y para reducir el tamaño de un programa. Las funciones se declaran asociadas a un tipo de valor “type”. Este valor será el que devolverá la función, por ejemplo int se utilizará cuando la función devuelve un dato numérico de tipo entero. Si la función no devuelve ningún valor entonces se colocará delante la palabra “void”, que significa “función vacía”. Después de declarar el tipo de dato que devuelve la función se debe escribir el nombre de la función y entre paréntesis se escribirán, si es necesario, los parámetros que se deben pasar a la función para que se ejecute.
\begin{lstlisting}
type nombreFuncion(parametros)
{
instruccion;
}
\end{lstlisting}
La función siguiente devuelve un número entero, delayVal() se utiliza para poner un valor de retraso en un programa que lee una variable analógica de un potenciómetro conectado a una entrada de Arduino. Al principio se declara como una variable local, v recoge el valor leído del potenciómetro que estará comprendido entre 0 y 1023, luego se divide el valor por 4 para ajustarlo a un margen comprendido entre 0 y 255, finalmente se devuelve el valor v y se retornaría al programa principal. Esta función cuando se ejecuta devuelve el valor de tipo entero v.
\begin{lstlisting}
int delayVal()
{
int v;                  	// crea una variable temporal v
v= analogRead(pot);  	// lee el valor del potenciometro
v = 4;            	// convierte 0-1023 a 0-255
return v;            	// devuelve el valor final
}
\end{lstlisting}
\section{\{\} entre llaves}
Las llaves sirven para definir el principio y el final de un bloque de instrucciones. Se utilizan para los bloques de programación setup(), loop(), if.., etc.
\begin{lstlisting}
type funcion()
{
instrucciones;
}
\end{lstlisting}

Una llave de apertura "\{“ siempre debe ir seguida de una llave de cierre “\}”, si no es así el programa dará errores.\\
El entorno de programación de Arduino incluye una herramienta de gran utilidad para comprobar el total de llaves. Sólo tienes que hacer click en el punto de inserción de una llave abierta e inmediatamente se marca el correspondiente cierre de ese bloque (llave cerrada).
\section{; punto y coma}
El punto y coma “;” se utiliza para separar instrucciones en el lenguaje de programación de Arduino. También se utiliza para separar elementos en una instrucción de tipo “bucle for”.
\begin{lstlisting}
int x = 13;   /* declara la variable x como tipo entero de valor 13 */
\end{lstlisting}
\textbf{Nota}: Olvidarse de poner fin a una línea con un punto y coma producirá en un error de compilación. El texto de error puede ser obvio, y se referirá a la falta de una coma, o puede que no. Si se produce un error raro y de difícil detección lo primero que debemos hacer es comprobar que los puntos y comas están colocados al final de las instrucciones.

\section{/*… */ bloque de comentarios}
Los bloques de comentarios, o comentarios multi-línea son áreas de texto ignorados por el programa que se utilizan para las descripciones del código o comentarios que ayudan a comprender el programa. Comienzan con / * y terminan con * / y pueden abarcar varias líneas.
\begin{lstlisting}
/* esto es un bloque de comentario no se debe olvidar
cerrar los comentarios estos deben estar equilibrados */
\end{lstlisting}
Debido a que los comentarios son ignorados por el compilador y no ocupan espacio en la memoria de Arduino pueden ser utilizados con generosidad. También pueden utilizarse para comentar bloques de código con el propósito de anotar informaciones para depuración y hacerlo mas comprensible para cualquiera.\\
\textbf{Nota}: Dentro de una misma línea de un bloque de comentarios NO se puede escribir otro bloque de comentarios (usando /*..*/).

\section{// línea de comentarios}
Una línea de comentario empieza con // y terminan con la siguiente línea de código. Al igual que los comentarios de bloque, los de línea son ignoradas por el programa y no ocupan espacio en la memoria.
\begin{lstlisting}
// esto es un comentario
\end{lstlisting}
Una línea de comentario se utiliza a menudo después de una instrucción, para proporcionar más información acerca de lo que hace ésta o para recordarla más adelante.