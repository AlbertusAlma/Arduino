\chapter{Aritmética}

Los operadores aritméticos que se incluyen en el entorno de programación son: suma, resta, multiplicación y división. Estos devuelven la suma, diferencia, producto o cociente (respectivamente) de dos operandos.
\begin{lstlisting}
y = y + 3;
x = x - 7;
i = j * 6;
r = r / 5;
\end{lstlisting}
La operación se efectúa teniendo en cuenta el tipo de datos que hemos definido para los operandos (int, long, float, etc..), por lo que, por ejemplo, si definimos 9 y 4 como enteros int, 9 / 4 devuelve de resultado 2 en lugar de 2,25 ya que el 9 y 4 son valores de tipo entero int (enteros) y no se ignoran los decimales con este tipo de datos.\\\\
Esto también significa que la operación puede sufrir un desbordamiento si el resultado es más grande que lo que puede ser almacenada en el tipo de datos. Recordemos el alcance de los tipos de datos numéricos que ya hemos explicado anteriormente.\\\\
Si los operandos son de diferentes tipos, para el cálculo se utilizará el tipo más grande de los operandos en juego. Por ejemplo, si uno de los números (operandos) es del tipo float y otra de tipo integer, para el cálculo se utilizará el método de float es decir, el método de coma flotante.\\\\
Elija el tamaño de las variables de tal manera que sea lo suficientemente grande como para que los resultados sean lo precisos que usted desea. Para las operaciones que requieran decimales utilice variables tipo float, pero sea consciente de que las operaciones con este tipo de variables son más lentas a la hora de realizarse el computo.\\
\textbf{Nota}: Utilice el operador (int) para convertir un tipo de variable a otro (también llamado casting) sobre la marcha. Por ejemplo, i = (int) 3,6 establecerá i igual a 3.
\section{Asignaciones compuestas}

Las asignaciones compuestas combinan una operación aritmética con una variable asignada. Estas son comúnmente utilizadas en los bucles tal como se describe más adelante. Estas asignaciones compuestas pueden ser:
\begin{lstlisting}
x++     // igual que x = x +1, o incremento de x en +1
x--    // igual que x = x - 1, o decremento de x en -1
x += y    // igual que x = x + y, o incremento de x en +y
x -= y    // igual que x = x - y, o decremento de x en -y
x *= y    // igual que x = x * y, o multiplica x por y
x /= y    // igual que x = x / y, o divide x por y
\end{lstlisting}
\textbf{Nota}: Por ejemplo, x * = 3 hace que x se convierta en el triple del antiguo valor x y por lo tanto x es reasignada al nuevo valor.
\section{Operadores de comparación}

Las comparaciones de una variable o constante con otra se utilizan con frecuencia en las estructuras condicionales del tipo if.. para testear si una condición es verdadera. En los ejemplos que siguen en las próximas páginas se verá su utilización práctica usando los siguientes tipo de condicionales:
\begin{lstlisting}
x == y        // x es igual a y
x != y        // x no es igual a y
x < y        // x es menor que y
x > y        // x es mayor que y
x <= y        // x es menor o igual que y
x >= y        // x es mayor o igual que y
\end{lstlisting}
\newpage{}
\section{Operadores lógicos}

Los operadores lógicos son usualmente una forma de comparar dos expresiones y devolver un VERDADERO o FALSO dependiendo del operador. Existen tres operadores lógicos, AND (\&\&), OR (||) y NOT (!), que a menudo se utilizan en estamentos de tipo if:\\
Logica AND:\\
\begin{lstlisting}
if (x > 0 && x < 5)    // cierto solo si las dos expresiones
                       // son ciertas
\end{lstlisting}
Logica OR:\\
\begin{lstlisting}
if (x > 0 || y > 0)    // cierto si una cualquiera de las
                       // expresiones es cierta
\end{lstlisting}
Logica NOT:\\
\begin{lstlisting}
if (!x > 0)        // cierto solo si la expresion es
                   // falsa
\end{lstlisting}